\documentclass{beamer}

\usepackage{listings}
\usepackage{color}

% Code colorature
\definecolor{paszt}{RGB}{252,252,252}
\definecolor{keret}{RGB}{220,220,220}
\lstset{
backgroundcolor=\color{paszt},
% showlines=true,
framexleftmargin=4mm,
framexrightmargin=4mm,
framextopmargin=2mm,
framexbottommargin=2mm,
frameround=tttt,
frame=trbl,
rulecolor=\color{keret}
}

\usetheme{Copenhagen}
\useinnertheme{rectangles}

% ---- Mongo theme ----
%\definecolor{light-background}{RGB}{210,250,170}
%\definecolor{dark-background}{RGB}{128,92,64}
%\usecolortheme[RGB={220,250,180}]{structure}

% ---- Vanilla theme ----
% \definecolor{light-background}{RGB}{250,250,190}
% \definecolor{dark-background}{RGB}{128,92,64}
% \usecolortheme[RGB={210, 210, 140}]{structure}

% ---- Blue theme ----
%\definecolor{light-background}{RGB}{200,220,240}
%\definecolor{dark-background}{RGB}{100,110,120}
%\usecolortheme[RGB={180, 200, 230}]{structure}

% ---- Green theme ----
\definecolor{light-background}{RGB}{229,237,204}
\definecolor{dark-background}{RGB}{156,163,140}
\usecolortheme[RGB={180, 210, 150}]{structure}

\setbeamercolor{palette primary}{fg=black, bg=light-background}
\setbeamercolor{palette quaternary}{fg=white,bg=dark-background}

\setbeamercolor{title}{fg=black}
\setbeamercolor{frametitle}{fg=black}

% Set font
%\usefonttheme{structurebold}

\frenchspacing

% Language packages
\usepackage[utf8]{inputenc}
\usepackage[T1]{fontenc}
\usepackage[magyar]{babel}

% AMS
\usepackage{amssymb,amsmath}

% Graphic packages
\usepackage{graphicx}

% Syntax highlighting
% \usepackage{listings}

%\usepackage{tikz}

%\begin{figure}[htb]
%\begin{center}
%	\includegraphics[scale=0.4]{ps_times.png}
%\end{center}
%\end{figure}

% ==============
\begin{document}
% ==============

\title[Interaktív közlekedési modell]{
{\Large Interaktív közlekedési modell procedurális generálása és megjelenítése WebGL segítségével}
}
\author[Kása Dániel Zoltán]{\Large Kása Dániel Zoltán}
\date{Miskolci Egyetem, 2019. június 17.}

% --------------------
% Title page
\frame{\titlepage}

% --------------------
\begin{frame}[fragile]
\frametitle{Közelekési szimulációk}

\textbf{Hasonló célú szoftverek}

\begin{itemize}
\item Road Traffic Library
\item SUMO
\end{itemize}

Külön dián lehetnek hozzá képek is.

\bigskip

\textbf{GeoSpatial adatbázisok}

Modellek bemutatása

\bigskip

A cél/feladat bemutatása

\end{frame}

% --------------------
\begin{frame}[fragile]
\frametitle{SUMO}

Egy kép az alkalmazásról (ami a dolgozatban is van)

\end{frame}

% --------------------
\begin{frame}[fragile]
\frametitle{A matematikai modell elemei I.}

Út, egyirányú út

Kereszteződés

\end{frame}

% --------------------
\begin{frame}[fragile]
\frametitle{Kereszteződés ábra}


\end{frame}

% --------------------
\begin{frame}[fragile]
\frametitle{A matematikai modell elemei II.}

Járművek

Épületek, környezeti elemek

\end{frame}

% --------------------
\begin{frame}[fragile]
\frametitle{Úthálózat generálása}

A generáló algoritmust kellene leírni

5-6 pontba leírni a fő lépéseket.
Az indoklást és részleteket elég lehet majd szóban mondani.

\end{frame}

% --------------------
\begin{frame}[fragile]
\frametitle{Unity WebGL}

Kifejezetten a fejlesztőkörnyezet bemutatása

\end{frame}

% --------------------
\begin{frame}[fragile]
\frametitle{A generálás implementálása}

Az elkészült program
- struktúrája
- funkciói
- egyéb sajátosságai

\end{frame}

% --------------------
\begin{frame}[fragile]
\frametitle{Szimulációk}

- Paraméterezés részletezése

- DEMO

\end{frame}

% --------------------
\begin{frame}[fragile]
\frametitle{Összegzés}

Elért eredmények tételes felsorolása

- Definiáltam a közlekedési modell alapelemeit.
- Megterveztem egy algoritmust az úthálózat és a hozzá tartozó városi környezet generálásához.
- Megterveztem a grafikus megjelenítéshez szükséges szoftver.
- Unity/WebGL segítségével implementáltam.
- Teszteltem különféle paraméterezésű szimulációk esetében.

További fejlesztési lehetőségek/tervek

\end{frame}

% --------------------
\begin{frame}[fragile]
\frametitle{Hivatkozások}

\begin{itemize}

\item Road Traffic Library
\item SUMO
\item WebGL
\item Unity

\end{itemize}

\end{frame}

% --------------------
\begin{frame}[fragile]
    \frametitle{\ }

\begin{center}
\Large \textbf{Köszönöm szépen a figyelmet!}
\end{center}

\end{frame}

\end{document}
