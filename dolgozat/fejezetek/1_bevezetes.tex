\Chapter{Bevezetés}

Szakdolgozatom témájaként egy interaktív közlekedési modell elkészítését választottam. Ez a modell egy procedurálisan legenerált városból fog állni, amelyen megfigyelhető majd a járművek közlekedése. A várost procedurálisan generált dekoratív elemekkel is ellátom, amely így városkép tekintetében színes megjelenítést fog adni. 

Az alkalmazás WebGL segítségével internetes böngészőben fog futni \cite{webgl}.
A dolgozat, valamint a program forrásfájljainak verziókezelésére a GitHub-ot használom.
Maga a program implementációjában a közlekedés fizikai realisztikusságára törekszem, az objektumok mozgása valós fizikai motorral kerül kiszámolásra.

A dolgozat témaválasztását illetően nagyban befolyásolt a grafikai téma iránti érdeklődésem. Továbbá, a jelenleg is aktív fejlesztés alatt álló, viszonylag friss technológia, a WebGL és az általa adott lehetőség, hogy a böngészőben futó kód közvetlen hozzáférést kap a számítógép videókártyájához, szintén felkeltette érdeklődésemet. Szintén inspirált még döntésemben a tanulmányaim alatt grafikából megszerzett ismeretanyag.

Mivel \textit{Unity Engine}-ben való fejlesztéssel már korábban is foglalkoztam, és az képes WebGL platformon futtatható formába lefordítani a programot, ezért célszerűnek tünt a közlekedési szimuláció megírásához is azt használnom.

A program felhasználó általi kezelését egy külön felülettel fogom biztosítani, amelyen az egyes paraméterek állíthatóak, ezzel nyújtva a program interaktív mivoltát.

Szakdolgozatomat az előbbiek megvalósításának lépéseit tekintve 5 részre tagoltam. Az első részben utánanézek az ebben a témában készült eddigi szoftvereknek, azok működését és felépítését fogom vizsgálni. 
A következő részben elkészítem a várost, és a rajta történő közlekedést leíró alapvető matematikai modellt. 
A harmadik részben a procedurális városgenerálás algoritmusát mutatom be, először annak vázlatos működését \textit{HTML Canvas} elemen szemléltetve, majd \textit{Unity Engine}ben a teljes működést. 
A következő rész a szimuláció tervezésével foglalkozik. Ebben részletesebben kitérek a keretrendszerben való fejlesztés lépéseire, magának a motornak a működésére. Ezen kívül természetesen részletezem a program implementációját, a szkriptekből kiemelt kódrészletek segítségével.
Az utolsó részben bemutatom az elkészült szimuláció működését, szemléltetem a szimuláció egyes elemeinek változását a paraméterek változtatásakor.

A modell megalkotása során nem törekszem a valóságban előforduló minden lehetséges közlekedési helyzet lefedésére. A hangsúlyt inkább a városkép kialakítására helyezem, olyan elemekkel díszítve mint a park, amely fákat és szökőkutat tartalmaz, többféle épület amelyek a városban való pozíciójuktól függnek.