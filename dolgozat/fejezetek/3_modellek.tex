\Chapter{Matematikai modell részletezése}

- Modellezési probléma specifikálása
- Pontosan milyen elemekből kellene felépíteni az adott város/térkép közlekedését?
- Mi lenne a modell célja?
- Irányított élek/egyirányú utak, többsávos utak modellje.

\Section{Modell specifikálása}
A közlekedésre irányuló modell fő tulajdonságai közé kell hogy tartozzon az elegendő információ tárolása ahhoz, hogy alapvetően tudjanak a szimulált 
járművek rajta közlekedni. Ezt rengeteg féle képpen meg lehet valósítani, sokféle eleme lehet egy adott térképnek, akár minden egyes előforduló úttípusra, 
szabályra, táblára tekinthetünk úgy mint a modell egyedi része. Ez a megközelítés viszont lehetségesen túlbonyolítaná a modellt, ezáltal a procedurális generálás
nem adna annyira elfogadható úthálózatot, lehetségesen életszerűtlen helyzetek épülhetnek fel a sok elem megléte, azok elhelyeződése miatt. Sokkal inkább célszerű a 
modell részeként tekinteni a főbb építőelemeket, melyek elegendőek magához az úthálózat generálásához, valamint néhány alapvető szabályt megvalósító elemet, mint például 
a lámpás útkereszteződés, az egyirányú út, valamint a többsávos út.
\Section{A modellben használt elemek}
\subsection{Egyenes út}
Alapvető egyenes útszakasz, ennek lehet több paramétere is. Ennek az elemnek elsősorban tartalmaznia kell a rá vonatkozó sebességhatárt, valamint az úton létező sávok számát.
Egyszerűen jellemezhető az elem a kezdő és a végpontjával, ugyanis az út szélessége annak paramétereiből (sávok száma) már adódik.
\subsection{Egyirányú út}
Hasonló az egyenes úthoz, viszont egyértelműen meg kell határozni a két végpont közül melyik a kiindulási, melyik a célpont. Ugyan azok a megkötések vonatkozhatnak rá, mint az egyenes
út szakaszra, azaz sebességkorlát, valamint sávok száma.
\subsection{Kanyar}
Vehető tulajdonképpen külön elemnek, de megvalósításában akár sok egymást követő egyenes szakasz, melyek közt kis folyamatos közelítések vannak a kanyar ívére. Ezen esetben elég
a kanyar kezdő és végpontját, valamint a használt egyenes szakaszok számát megadni annak jellemzésére. Alternatívaként a kanyar szöge is felhasználható jellemzésre. Különleges eset
ha a kanyar egy egyirányú úton történik, ilyenkor az egyetlen változtatás az, hogy a kanyar közelítésére használt útszakaszoknak egyirányúnak kell lenniük.
\subsection{Útkereszteződés}
Három vagy több út találkozásánál használatos elem. Tulajdonságai közé tartozhat az hogy lámpás útkereszteződés, vagy sem, valamint ha nem lámpás akkor van-e megállási kötelezettség.
Lehetséges valamint egyszerű esetben, azaz betorkolló útnál eljelölni a felülrendelt és alárendelt utat is. Az útkereszteződés jellemezhető annak középpontjával, valamint az azt érintő utak
halmazával.
\subsection{Személygépjármű}
Legalapvetőbb közlekedési jármű. Az autókról nyilván kell tartani azok jelenlegi pozícióját, ebbe beleértve hogy melyik objektumon tartózkodnak éppen, és annak pontosan milyen pontján. Továbbá azt
is hogy merre tartanak éppen, lokálisan a jelenlegi útszakasz végpontját, globálisan pedig a városban azon csomópont, ahova el szeretne jutni. Ezen elem jellemzői közé tartozik a sebessége, valamint többsávos úton 
hanyadik sávban közlekedik éppen. Tudnia kell a vele egy úton közlekedő járművekől is, azok pozíciójától függ a viselkedése. A járművek kezdetben véletlenszerű helyen kezdenek, majd ha sikeresen eljutottak a célpontjukhoz 
eltűnnek a modelből (leparkolt kocsinak tekintve). Paraméterezést tekintve ilyenkor új jármű jelenik meg ha jelenleg kevesebb van mint a beállított maximális érték.
\subsection{Autóbusz}
Kissé különlegesebb járműtípus, vonatkozik rá néhány további szabály. Többek között elsőbbsége van megállóból elindulva, útkereszteződésnél nagy ívben kell fordulnia, tehát többsávos út esetén ilyenkor
beljebb kell venni. A buszmegállók között megkötött útvonalon kell haladnia, mindegyiknél pedig meg kell állnia. A szimuláció teljes ideje alatt az útvonalat követi, kezdetben egyik véletlenül kijelölt megállóból indul.
\subsection{Buszmegálló}
Autóbusznak elhelyezett megállási pont. Útkereszteződésen kívül bármely csomópont eljelölhető buszmegállónak. Szükséges információ a buszmegállók közötti minimum távolság, valamint az összes buszmegálló száma.
Ezek az adatok paramatérezhetőek is. Köztük az útvonal kiszámítása történhet egy véletlenszerűen választott kezdőponttól a kialakult gráf mohó algoritmussal történő bejárásával.
\subsection{Épületek}
Az utak között kialakult üres területek feltöltésére szolgáló objektumok. Jellemzői közé tartozik az épület négy csúcsa, valamint a magassága. Az épület pozíciójának valamint méreteinek meghatározása a változatosság érdekében 
véletlenszerű.
\Section{A modell célja}
\subsection{Generálási szempont}
Magának a városnak, úthálózatnak a generálására vonatkozóan a következők a célok:
\begin{itemize}
\item Viszonylag kevés elemből álljon össze a generált város
\item Ne legyen életszerűtlen az összeállított úthálózat
\item Tartalmazza a felvetett közlekedési helyzetek szimulálására szükséges elemeket
\item Vizuálisan emlékeztessen egy átlagos városra
\item Bizonyos mértékig paraméterezhető legyen, főleg a méreteket érintve
\end{itemize}
\subsection{Szimulációs szempont}
A már legenerált úthálózaton a szimulációra tekintve ezek a célok:
\begin{itemize}
\item A járművek kövessék a modellben definiált alapvető szabályokat
\item A szimuláció bizonyos mértékben paraméterezhető legyen, főleg a járművek számát illetően
\item Reagáljanak egymásra a járművek
\item Ne legyen életszerűtlen a járművek viselkedése
\item Teljesítmény szempontjából elfogadható legyen egy adott méretig
\item Felhasználható statisztikai eredményeket le lehessen menteni
\end{itemize}