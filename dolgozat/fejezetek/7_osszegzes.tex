\Chapter{Összegzés}

A feladat teljesítését természetesen az első részben leírt szoftverek vizsgálatával kezdtem, melyből megtudtam pár információt az ilyen jellegű szoftverek működéséről. Az ott leírtak összefoglalják, hogy egy átlagos ilyen célú szoftver milyen felépítéssel bír, általánosan milyen funkcióik vannak, valamint az általuk kezelt objektumok milyen tulajdonságokból épülnek fel.

A modell megalkotásánál definiált kritériumokat teljesítettem, az elkészült programban megjelennek az ott leírt elemek, forgalmi szituációk, azok működése a leírtak szerint történik.

A generálás folyamatát a harmadik részben dokumentáltam. Az algoritmus eredményéből látszik, hogy alkalmas a forgalmi helyzetek megjelenítésére, az átlagos közlekedésre. Városkép szempontjából elfogadható eredményt ad, az épületek és más díszítőelemek logikusan helyezkednek el a város területén.

A szimuláció tervezésével foglalkozó fejezetben részletesen dokumentáltam az egyes szkriptek működését, valamint a Unity Engine-en belüli implementációt. Az egyes algoritmusokat kódpéldákkal is szemléltettem. Szimuláció közben látható ennek az implementációnak az eredménye, a járművek mozgása a Unity fizikai motorján keresztül történik.

Az utolsó részben megvizsgáltam a paraméterezéssel kapcsolatban felállított követelmények teljesülését. A program a megalkotott felületen keresztül működésében befolyásolható, a különböző paraméterezések eredménye pedig szemmel látható.

Összefoglalva, az elkészült alkalmazással szimulálhatók az egyszerűbb közlekedési szituációk, és ehhez a megjelenítésben a tipikus városi környezeti elemeket is láthatjuk. Az alkalmazás megfelelő alapot jelent a későbbi fejlesztésekhez. Az elkészült programkód könnyen bővíthető új elemekkel, például járművekkel, környezethez tartozó modellekkel, a város generálására vagy a közlekedési szimulációra vonatkozó szabályokkal.
